\documentclass[a4paper,12pt,oneside]{scrreprt}

\usepackage[T1]{fontenc}
\usepackage[utf8]{inputenc}
\usepackage[ngerman]{babel}

\usepackage[bookmarks,pagebackref,bookmarksnumbered]{hyperref}

\usepackage{amsmath}
\usepackage{amsfonts}
\usepackage{amssymb}

\begin{document}

\section*{Digitale Signaturen}

\textit{Ein Vortrag Rahmen des Seminars Kryptographie, gehalten von Bernhard Häussner am 17. Januar 2013 im Wintersemester 2012/13. }\\

Digitale Signaturverfahren werden benutzt, um digitale Dokumente kryptographisch vor Veränderungen zu schützen und ihre Authentizität zu belegen. 
Sie erfüllen einen ähnlichen Zweck wie Unterschriften auf Papier. Mit einem geheimen, privaten Signaturschlüssel erstellt Alice zu einer
Nachricht eine Signatur, die Bob mit ihrem öffentlichen Verifikationsschlüssel prüfen kann. 
Die Signatur gilt nur für die signierte, unveränderte Nachricht und aus ihr lässt sich
nicht auf den Signaturschlüssel schließen. 

\subsection*{RSA-Signatur}

Die Sicherheit der RSA-Signatur basiert auf der Sicherheit des RSA-Verschlüsselung-Verfahrens. 
Da für RSA gilt $m = E(D(m,d),e)$, kann für das Signieren die Verschlüsselung und für das Verifizieren die
Entschlüsselung verwendet werden. 

\paragraph{Schlüsselerzeugung und Vorbereitungen}

Alice wählt/setzt ebenso wie für die Verschlüsselung: $p, q$ prim, $n = pq$ (Modul), $e$ 
mit $1 < e < (p-1)(q-1)$, $\gcd(e,(p-1)(q-1)) = 1$ (öff. Schlüssel),
$d$ mit $1 < d < (p-1)(q-1)$ und $de \equiv 1$ mod $(p-1)(q-1)$ (priv. Schlüssel). 

\paragraph{Signatur-Erstellung}

Alice erstellt die Signatur $s$ der Nachricht $m \in \{0,1,\dots,n-1\}$ mit dem privatem Schlüssel $d$: 

\[ s = m^d \mod n \]

\paragraph{Verifikitation}

Bob verifiziert die Signatur $s$ mit dem öffentlichem Schlüssel $(n,e)$ und vergleicht die so erhaltene
Nachricht $m$ mit der gesendeten:

\[ m = s^e \mod n \]

\paragraph{Redundanz oder Hashfunktion}

Da sich aus zwei gültigen Signaturen durch Multiplikation eine dritte generieren lässt und auch
jede Zahl für irgendeine Nachricht eine gülitge Signatur darstellt, werden statt der Nachricht
eine gepaddede Nachicht (z.B. $m \circ m$) oder ein Hashwert signiert und die Nachricht wird gesondert
übertragen.  

\subsection*{Elgamal-Signatur}

Die Sicherheit der Elgamal-Signatur basiert auf der Schwierigkeit diskrete Logarithmen zu berechnen. 
Da hier Ver- und Entschlüsselung nicht vertauschbar sind, muss ein modifiziertes Verfahren verwendet
werden. Das Prinzip der Elgamal-Signaturen wurde als DSA mit festen Bitlängen und einem 
etwas effizienterem Verfahren standadisiert. 

\paragraph{Schlüsselerzeugung und Vorbereitungen}

Alice wählt/setzt ähnlich zur Verschlüsselung: $p$ prim, $h : \Sigma^* \to \{1,2,\dots,p-2\}$ 
kollisionsresistente Hashfunktion, $g \mod p$ Primitivwurzel, 
$a \in \{1,2,\dots,p-2\}$ zufällig (priv. Schlüssel), $A = g^a \mod p$.

\paragraph{Signatur-Erstellung}

Alice erstellt die Signatur $(r,s)$ der Nachricht $m \in \Sigma^*$ mit dem privatem Schlüssel $a$ 
und einer geheimen Zufallszahl $k \in \{1,2,\dots,p-2\}$ mit $\gcd(k,p-1) = 1$, sodass
$k^{-1}$ mod $p-1$ existiert:

\[ r = g^k \text{ mod } p \text{~~~~und~~~~} s = k^{-1}(h(m)-ar) \text{ mod } (p-1)\]

\paragraph{Verifikitation}

Bob verifiziert die Signatur $(r,s)$ mit dem öffentlichem Schlüssel $(p, g, A)$ durch:

\[ 1 \leq r \leq p-1 \text{~~~~und~~~~} A^r r^s \equiv g^{h(m)} \text{ mod } p\]

Duch Einsetzen der Terme zur Signaturerstellung kann man sich leicht von der Korrektheit
überzeugen. 

\subsection*{Lamport-Diffie-Einmal-Signatur}

Anders als bei den herkömmlichen Signaturverfahren basiert die Sicherheit des Lamport-Diffie-Einmal-Signaturverfahrens
nur auf der Nichtinvertierbarkeit einer beliebigen verwendeten Einwegfunktion, beispielsweise einer kollisionsresistenten Hashfunktion. 
Es wird praktisch selten verwendet, da für jede zu signierende Nachricht ein neues Schlüsselpaar erzeugt werden muss, 
ansosnten könnte der Schlüssel leicht rekonstruiert werden. Dieses Problem kann z.B. mit Hashbäumen gelöst werden. 
Um Nachrichten beliebiger Länge signieren zu können, wird eine Hashfunktion vorgeschaltet. 

\paragraph{Schlüsselerzeugung und Vorbereitungen}

Alice wählt den Sicherheitsparameter $k \in \mathbb{N}$ und eine Einwegfunktion 
$H : \{0,1\}^k \mapsto \{0,1\}^k$ und setzt außerdem für jeden Signaturvorgang:

\[ X = (x_{1,0},x_{1,1},x_{2,0},x_{2,1},\dots,x_{n,0},x_{n,1}) \in (\{0,1\}^k)^{2n} \text{(zufällig)}\]
\[ Y = (y_{1,0},y_{1,1},y_{2,0},y_{2,1},\dots,y_{n,0},y_{n,1}) \text{ mit } y_{j,j} = H(x_{i,j}) \text{(öff. Schlüssel)}\]

\paragraph{Signatur-Erstellung}

Alice erstellt aus der Nachricht $m = (m_1, m_2, \dots, m_n) \in \{0,1\}^n$ die Signatur $S$:

\[ S = (s_1, s_2, \dots, s_n) \text{ mit } s_i = x_{i,m_i}\]

\paragraph{Verifikitation}

Bob verifiziert die Signatur $S$ mit dem öffentlichem Schlüssel $Y$ druch:

\[ (H(s_1),H(s_2),\dots,H(s_i)) = (y_{1,m_1},y_{2,m_2},\dots,y_{n,m_n})\]

\end{document}

