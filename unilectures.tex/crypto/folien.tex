\documentclass[a4paper,12pt,oneside]{scrreprt}

\usepackage[T1]{fontenc}
\usepackage[utf8]{inputenc}
\usepackage[ngerman]{babel}

\usepackage[bookmarks,pagebackref,bookmarksnumbered]{hyperref}

\usepackage{amsmath}
\usepackage{amsfonts}
\usepackage{amssymb}

\begin{document}

\section*{RSA-Signatur}

\subsection*{Schlüsselerzeugung und Vorbereitungen}

Alice wählt/setzt ebenso wie für die Verschlüsselung:

\begin{itemize}
 \item $p, q$ prim. 
 \item $n = pq$. (Modul)
 \item $e$ mit $1 < e < (p-1)(q-1)$, $\gcd(e,(p-1)(q-1)) = 1$. (öff. Schlüssel)
 \item $d$ mit $1 < d < (p-1)(q-1)$ und $de \equiv 1$ mod $(p-1)(q-1)$. (priv. Schlüssel)
\end{itemize}


\subsection*{Signatur-Erstellung}

Alice erstellt Signatur $s$ der Nachricht $m \in \{0,1,\dots,n-1\}$ mit privatem Schlüssel $d$: 

\[ s = m^d \mod n \]

\subsection*{Verifikitation}

Bob verifiziert die Signatur $s$ mit dem öffentlichem Schlüssel $(n,e)$ und vergleicht die so erhaltene
Nachricht $m$ mit der gesendeten:

\[ m = s^e \mod n \]


\clearpage
\section*{Elgamal-Signatur}

\subsection*{Schlüsselerzeugung und Vorbereitungen}

Alice wählt/setzt ähnlich zur Verschlüsselung:

\begin{itemize}
 \item $p$ prim. 
 \item $h : \Sigma^* \to \{1,2,\dots,p-2\}$ kollisionsresistente Hashfunktion.
 \item $g \mod p$ Primitivwurzel.
 \item $a \in \{1,2,\dots,p-2\}$ zufällig. (priv. Schlüssel)
 \item $A = g^a \mod p$.
\end{itemize}

\subsection*{Signatur-Erstellung}

Alice erstellt Signatur $(r,s)$ der Nachricht $m \in \Sigma^*$ privatem Schlüssel $a$ 
und geheimer Zufallszahl $k \in \{1,2,\dots,p-2\}$ mit $\gcd(k,p-1) = 1$, sodass
$k^{-1}$ (mod $p-1$) exitiert:

\[ r = g^k \text{ mod } p \text{~~~~und~~~~} s = k^{-1}(h(m)-ar) \text{ mod } (p-1)\]

\subsection*{Verifikitation}

Bob verifiziert die Signatur $(r,s)$ mit öffentlichem dem Schlüssel $(p, g, A)$ durch:

\[ 1 \leq r \leq p-1 \text{~~~~und~~~~} A^r r^s \equiv g^{h(m)} \text{ mod } p\]


\clearpage
\section*{Lamport-Diffie-Einmal-Signatur}

\subsection*{Schlüsselerzeugung und Vorbereitungen}

Alice wählt:

\begin{itemize}
 \item $k \in \mathbb{N}$. (Sicherheitsparameter)
 \item $H : \{0,1\}^k \mapsto \{0,1\}^k$ Einwegfunktion. 
\end{itemize}

und außerdem für jeden Signaturvorgang:

\begin{itemize}
 \item $X = (x_{1,0},x_{1,1},x_{2,0},x_{2,1},\dots,x_{n,0},x_{n,1}) \in (\{0,1\}^k)^{2n} $ zufällig. (priv. Schlüssel)
 \item $Y = (y_{1,0},y_{1,1},y_{2,0},y_{2,1},\dots,y_{n,0},y_{n,1}) $ mit $y_{j,j} = H(x_{i,j})$. (öff. Schlüssel)
\end{itemize}

\subsection*{Signatur-Erstellung}

Alice erstellt die Signatur $S$ aus der Nachricht $m = (m_1, m_2, \dots, m_n) \in \{0,1\}^n$:

\[ S = (s_1, s_2, \dots, s_n) \text{ mit } s_i = x_{i,m_i}\]

\subsection*{Verfikitation}

Bob verifiziert die Signatur $S$ mit dem öffentlichem Schlüssel $Y$ durch:

\[ (H(s_1),H(s_2),\dots,H(s_i)) = (y_{1,m_1},y_{2,m_2},\dots,y_{n,m_n})\]

\end{document}

