\chapter{Einleitung}\label{chap:einl}

Das Problem der Tourenplanung f�r mehrere Fahrzeuge ist als das \emph{Vehicle Routing Problem} (VRP) bekannt und wurde 1959 von Dantzig \& Ramser~\cite{dr-tdp-59} eingef�hrt. Dabei soll eine Menge von Auftr�gen f�r eine Menge von Kunden zu optimalen Touren (im Sinne von Kosten und Wartezeiten) f�r Fahrzeuge zusammengestellt werden. Eine Tour besteht dabei aus einer Teilmenge der gegebenen Auftr�ge und einer dazugeh�rigen Reihenfolge. Dieses Problem wurde in der Vergangenheit bereits ausgiebig behandelt. Als Verallgemeinerung des \emph{Traveling Salesman Problems} (TSP) ist auch das VRP NP-schwer. Daher wurden mehrere Heuristiken entwickelt, die mit geringem Rechenaufwand und kurzer Laufzeit gute, aber nicht notwendigerweise optimale L�sungen f�r das Problem berechnen. F�r dieses Problem existieren diverse Anwendungsgebiete, zum einen im Logistikbereich, zum Anderen aber auch bei allen Unternehmen, die ihre Kunden beliefern (z.B. Getr�nkelieferanten, M�belindustrie oder Zeitungsverlage). Eine bekannte Variante des VRP ist das $m$-TSP, bei dem genau $m$ Routen konstruiert werden sollen, um alle gegebenen Punkte zu besuchen. Im Gegensatz zu diesem ist beim VRP die Anzahl der Touren nicht fest vorgegeben, sondern nur durch die Anzahl der verf�gbaren Fahrzeuge nach oben beschr�nkt und soll minimiert werden.


\section{Problemdefinition}\label{sec:prob}

\section{Bisheriger Stand der Forschung}\label{sec:forsch}

Die im Sinne von G�te und Laufzeit besten Heuristiken zur L�sung des VRPTW wurden von Solomon~\cite{s-vrsptw-87} gegen�bergestellt. Dabei verglich er folgende Algorithmen miteinander: die Savings-Heuristik von Clarke und Wright~\cite{bq-ipdp-64}, eine Nearest-Neighbor Heuristik, die Insertion-Heuristik (auch I1 genannt) und eine Sweep-Heuristik von Gillet und Miller~\cite{gm-havdp-74}. Die besten Ergebnisse erzielte er dabei mit der Insertion-Heuristik. Der popul�rste Algorithmus ist allerdings der Savings-Algorithmus. Von Paessens~\cite{p-savrp-88} wurden Verbesserungen der Zielfunktion genauer betrachtet sowie eine prozedurale Implementierung angegeben. 

Br�ysy \& Gendreau~\cite{s-bp-05} setzten sich mit verschiedenen M�glichkeiten der lokalen Suche und Nachbearbeitung auseinander. Dabei gaben sie zuerst einen �berblick �ber verschiedene Swap-Verfahren, bei denen einzelne Kunden oder auch ein Folge von Kunden entweder innerhalb der Tour, oder auch zwischen verschiedenen Touren getauscht werden. Anschlie�end wurden verschiedene Heuristiken miteinander verglichen. Dabei schnitten das Ruin \& Recreate Verfahren von Schrimpf et al.~\cite{sssd-rrp-00}, eine Greedy-Such-Heuristik von Prosser und Shaw~\cite{ps-sgs-96} und eine deterministische, alternierende k-OPT Heuristik von Cordone und Wolfer-Calvo~\cite{lmt-eavrpst-01} am besten ab.

Laporte~\cite{l-vrpeaa-92} besch�ftigte sich intensiv mit exakten L�sungsverfahren. Dabei stellte er mehrere L�sungswege vor, die haupts�chlich auf ganzzahliger linearer Programmierung basieren. Laporte et al.~\cite{NET:NET3230160104} benutzten daf�r einen Branch-and-Bound Algorithmus, der auf dem Spezialfall m-TSP des VRP beruht. Christofides et al.~\cite{lmt-eavrpst-01} bauten ihren Algorithmus sogar auf dem Spezialfall k-degree center tree des m-TSP auf, w�hrend Balinski und Quandt~\cite{bq-ipdp-64} eine Mengen Partitionierungs Formulierung des VRP entwickelten. Laporte et al.~\cite{c-tatsp-85} arbeiteten eine 2-Index Fluss Formulierung f�r Kapazit�ts- und Distanzschranken aus, Fisher und Jaikumar~\cite{fj-gahvr-81} benutzten eine 3-Index Fluss Formulierung f�r das VRP mit Kapazit�tsbeschr�nkung und Zeitfenstern, aber ohne Stoppzeiten.

\section{Zielstellung}\label{sec:ziel}

\section{Durchf�hrung}\label{sec:durchf}